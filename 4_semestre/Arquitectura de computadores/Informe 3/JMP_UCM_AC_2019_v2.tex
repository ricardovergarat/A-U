% !TeX spellcheck = es_ES
\documentclass[10pt,a4paper]{article}
\usepackage[utf8x]{inputenc}
\usepackage[left=2cm, right=2cm, top=1.50cm, bottom=1.50cm]{geometry}
\usepackage{amsmath}
\usepackage{amsfonts}
\usepackage{amssymb}
\usepackage{graphicx}
\usepackage{theorem}
\usepackage{subfig}
\usepackage{epstopdf}
\usepackage[hyphens]{url}
\newcommand{\Hi}{$\mathcal{H}_{\infty}$}
\usepackage{url}
\theorembodyfont{ \normalfont }
\newtheorem{theorem}{\it Theorem}
\usepackage{hyperref}


%%%%%%%%%%%%%%%%%%%%%%%%%%%%%%%%%%%%%%%%
%%%%%%%%%%%%%%%%%%%%%%%%%%%%%%%%%%%%%%%%
%\usepackage[utf8]{inputenc}
%\usepackage[T1]{fontenc}
\usepackage{lmodern}
\usepackage{times}
\usepackage[portuges, english, brazil]{babel}
%\usepackage[english]{babel}
\usepackage{graphicx}
\usepackage[hyphens]{url}
\usepackage{epsfig}
\usepackage{psfrag}
\usepackage{fancybox}
\usepackage{color}
\usepackage{amsmath}
\usepackage{multirow}
\usepackage{amssymb}
\usepackage{latexsym}
\usepackage{amsmath}
\usepackage{balance}
\usepackage{times}
%\usepackage{++ure}
\usepackage{theorem}
\usepackage{pifont}
\usepackage{tabularx}
\usepackage{textcomp}
\usepackage{colortbl}
\usepackage{cite}
\usepackage{icomma}
\usepackage{bbold}
\usepackage{dsfont}
\usepackage{amssymb}
\usepackage[usenames,dvipsnames]{xcolor}

\usepackage{mathptmx}

%\usepackage{algorithm}
%\usepackage[noend]{algpseudocode}
\usepackage{subfig}
\usepackage{epstopdf}

\newcommand{\I}{\mbox{I}}
\newcommand{\Hd}{$\mathcal{H}_{2}$}
%\newcommand{\Hi}{$\mathcal{H}_{\infty}$}
\newcommand{\Tr}{\mbox{ Tr}~}
\newcommand{\G}{$\mathcal{G}$}
\newcommand{\K}{\mathcal{K}}
\newcommand{\Y}{\mathcal{Y}}
\newcommand{\Z}{\mathcal{Z}}
\newcommand{\X}{\mathcal{X}}
\newcommand{\Sf}{\mathcal{S}}
\newcommand{\Kb}{\mathds{K}}

\newcommand{\A}{\mathcal{A}}
\newcommand{\C}{\mathcal{C}}

\newcommand{\Kset}{\mathcal{K}}
\newcommand{\Pset}{\mathcal{P}}
\newcommand{\Rset}{\mathcal{R}}
\newcommand{\KKset}{\Gamma}
\newcommand{\Uset}{\mathcal{U}}
\newcommand{\UUset}{\Lambda}
\def\real{\mathds{R}}
\def\nat{\mathds{N}}
\def\sym{\mathds{S}}
\def\integer{\mathds{Z}_{+}}
\def\ds{\displaystyle}

%%%%%%%%%%%%%%%%%%%%%%%%%%%%%%%%%%%%%%%%%%%%%%%%
%%%%%%%%%%%%%%%%%%%%%%%%%%%%%%%%%%%%%%%%%%%%%%%%






\begin{document}
%\begin{flushleft}
%Examen del aula \textit{Control De Procesos Industriales},
%segundo  2018. Lunes 17 de diciembre 8:15pm a 10:15pm.  %\href{https://www.researchgate.net/profile/Jonathan_Palma2}{[Research Gate]}.
%\end{flushleft}
\vspace{6 mm}


\begin{center}
\textbf{{\LARGE   Arquitectura de Computadores\\ 
		\vspace{2 mm}	
			  {\em Costo Comunicacional}} } 
\end{center}
\vspace{3 mm}		
\textbf{Profesor: Jonathan M. Palma}~\\ %~(Profesor)}


	
	
\section{Resumen del Modulo}
En el diseño de soluciones computacionales existen dos formas de afrontar los problemas que comprenden realizar cálculos en gran volumen.    El primero {por hardware} permite realizar cálculos complejos con gran eficiencia  pero las intrusiones utilizadas requieren  hardware dedicado. 
Por contra parte, las soluciones {\em por software} utilizan un set resumido de instrucciones las cuales  combinan, permitiendo realizar cálculos mas complejos. 
 El presente modulo tiene por objetivo exponer las ventas y complicaciones de cada estrategia para   abordar la resolución de cálculos complejos en procesadores de aritmética limitada y computadores convenciones de escritorio. 
  

\section{Evaluación}


Contenidos del Informe 

\textbf{Sección 1 ::  Introducción.}

3 párrafos

- Párrafo de texto uno, Marco teórico general "Que es el costo computacional y por que es importe para informática".

- Párrafo de texto dos,  Conceptualización del problema particular a tratar.

- Párrafo de texto tres, Explicar  caso de estudio. 

 $>>>$ {\em Realizar un texto coherente  entre secciones partiendo  de ideas generales hasta llegar a caso de estudio particular.}


\textbf{Sección 2 :: Problema.}

Mínimo 3 párrafos

Párrafo 1

- Definir el problema general conceptualización en detalle "calcular esfuerzo computacional al  resolver LMIs cuantificado en n\'umero de variables escalares y lineas de LMIs".

{\em Recordar} Problema a tratar es 

$$P>0 ~~ P \in \mathbb{R}^{n_x \times n_x}$$

$$ A(\theta(k))'P + P A(\theta(k))<0  \forall   \theta(k) $$

Párrafo 2

- Desarrollar una metodología para crear los vértices de  $A(\theta(k)) \in \mathbb{R}^{n_x \times n_x}$ para un $n_x$ arbitrario.

Párrafo 3

- Describir los recursos computaciones utilizados en el trabajo. 

\textbf{Sección 3 :: Resultados.}

Dado la estructura descrita en la sección anterior para   $A(\theta(k))$ graficar  el n\'umero variables escalares y lineas de LMIs en función del orden de $A(\theta(k))$ ($n_x$).

Crear una base de datos con 100 sistemas estables para $n_x=3,4,...10$ y mostrar  el tiempo medio de resolución y desvió estándar obtenido para  cada grado $n_x$.

{\em Opcional (Puntos Extras): A) utilizar mas de un computador para estudiar como varia el tiempo de resolución de las LMIs en función del hardware utilizado. B) comparar con la base de datos de compañeros.}


\textbf{Sección 4 :: Conclusión}

2 párrafos;

Primero. Conclusión General

Segundo. Alguno de estos tópicos; trabajo futuros, problemas no abordados, comentarios relevantes etc. 


\textbf{Observaciones}

-Trabajo debe tener mínimo 3 paginas y un máximo 4 paginas en formato IEEE doble columna.

 \url{https://journals.ieeeauthorcenter.ieee.org/create-your-ieee-journal-article/authoring-tools-and-templates/ieee-article-templates/}.

-6 referencias mínimo que sean parte del texto.


%
%
%
%
%
%
%\color{blue}
%Días de Aula.
%\begin{itemize}
%\item \textbf{Viernes 18 de Octubre en Bloque 6-7 y 8-9:} Introducción,  presentación de materiales y  evaluación, conceptos de optimizacion, Primera parte del Tutorial de LMI {\em linear matrix inequality}.
%\item \textbf{Miércoles  23 de Octubre en Bloque 1-2 y 3-4:} Programación en Matlab y conceptos de optimizacion en espacios vectoriales y costo computacional asociado, Segunda parte  del Tutorial de LMI {\em linear matrix inequality}. 
%\item \textbf{Viernes 25 de Octubre en Bloque 6-7 y 8-9:} Prueba pendiente Profesor Frenando.
%\item \textbf{Miercoles 30 de Octubre en Bloque  1-2 y 3-4:} Trabajo individual, envió informe preliminar.
%\item \textbf{Miércoles 06 de noviembre  Bloque 1-2 y 3-4:}  Finalización
% del modulo. 
%\end{itemize}
%

\subsection{Evaluación}

\begin{itemize}
	\item \textbf{Preliminar:} Opcional fecha de envió 5 Enero. 
	\item \textbf{Informe:} 10 de Enero. 
\end{itemize}

Basado en el Tutorial de LMI, trabajo individuales. 

%
%\subsection{Descripción de Actividades}	
%\subsection{Aula 1}
%Tópicos a tratar 
%\begin{itemize}
%	\item Soluciones por  {\em por software} y {por hardware}.
%	\item Instrucciones básicas en procesadores, problema de invertir matriz (Unidad de coma flotante vs Unidad aritmético lógica).  
%	\item Ejemplo GPU vs CPU. 
%	\item Desafíos en Inteligencia artificial y optimizan. 
%	\item Tarea evaluar el tiempo asociado a invertir una matriz en Arduino o Matlab. 
%\end{itemize}
%\subsection{Aula 2}
%\subsection{Aula 3}
%\subsection{Aula 4}
%\subsection{Aula 5}

\section{Referencias y Material Complementario}	

GPU  \url{https://hipertextual.com/archivo/2013/12/hardware-gpu-grafica/}\\
Unidad de coma flotante  \url{https://es.wikipedia.org/wiki/Unidad_de_coma_flotante}\\
Suma complemento a 2 \url{https://www.youtube.com/watch?v=4bO7Cx8qKfI}\\
Complemento a 2  \url{https://www.youtube.com/watch?v=umCFglE_eFU}\\

%\bibliographystyle{alpha}
%\bibliography{listaabreviada,JMPref_20181107_ql,estgeral,hinf,filtering,hoofilt,robust,fldp,saida,switched,estquad,markov,sdp,fuzzy,fragile,satcon,estab,ltv,polinomial,scheduling,bmi,ncs,mpc,delay,sispot}

\end{document}